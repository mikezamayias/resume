% Copyright (c) 2022 Mike Zamayias All rights reserved.
% Use of this source code is governed by a BSD-style
% license that can be found in the LICENSE file.

% !TEX TS-program = XeLaTeX
% !TEX encoding = UTF-8 Unicode
% !TEX spellverify = gr-GR

\documentclass[10pt]{article}
\usepackage[a4paper, margin=1cm]{geometry}
\usepackage{polyglossia}
\usepackage{fontspec}
\usepackage{moresize}
\usepackage{minted}
\usepackage{amsmath}
\usepackage{paracol}
\usepackage{tikz}
\usepackage{graphicx}
\usepackage{multicol}
\usepackage{xcolor}
\usepackage{fontawesome}
\usepackage[hidelinks]{hyperref}

% provides \isempty test
\usepackage{xstring, xifthen}

% debugging
% \usepackage{showframe}

% tikz stuff
\usetikzlibrary{shapes, backgrounds,mindmap, trees}

% polyglossia stuff
\setdefaultlanguage{greek}
\setotherlanguage{english}

% fontspec stuff
\newfontfamily\greekfont{JetBrains Mono}
\newfontfamily\greekfontsf{JetBrains Mono}
\setmainfont{JetBrains Mono}
\setsansfont{JetBrains Mono}
\setmonofont{FiraCode Nerd Font}

% paragraph stuff
\setlength{\parindent}{0.1cm}

% color stuff
\definecolor{light}{HTML}{F3F3F3}
\definecolor{dark}{HTML}{303030}
\definecolor{violet}{HTML}{646CFF}
\definecolor{blue}{HTML}{0078D4}
\definecolor{green}{HTML}{41B883}
\pagecolor{light}

% hyperref stuff
\hypersetup{
  colorlinks=true,
  linkcolor=violet,
  filecolor=blue,
  urlcolor=violet,
  pdfpagemode=FullScreen,
}

% returns minipage width minus three times \fboxsep
% to keep padding included in width calculations
% can also be used for other boxes / environments
\newcommand{\mpwidth}{\linewidth-3\fboxsep}

% centers text vertically in line: https://tex.stackexchange.com/a/17102
% param 1: any
\newcommand*{\vcenteredhbox}[1]{\begingroup
\setbox0=\hbox{#1}\parbox{\wd0}{\box0}\endgroup}

% base class to wrap any text based stuff here. Renders like a paragraph.
% Allows complex commands to be passed, too.
% param 1: *any
\newcommand{\cvtext}[1] {
	\begin{tabular*}{1\mpwidth}{p{0.98\mpwidth}}
		\parbox{1\mpwidth}{#1}
	\end{tabular*}
}

% Renders a a CV section headline with a nice underline in accent color.
% param 1: section title
\newcommand{\cvsection}[1] {
  \vspace{0.6cm}
	\cvtext{
    \textbf{\LARGE{\textcolor{dark}{#1}}}\\[-6pt]
    \textcolor{green}{ \rule{0.1\textwidth}{2.1pt} }\\
  }
  \\
}

% Renders a button.
% param 1: link
% param 2: icon
% param 3: icon size
\newcommand{\button}[3]{
  \href{#1}{\textcolor{violet}{\vcenteredhbox{#3#2}}}
}

% Renders an active social media button.
% param 1: link
% param 2: icon
\newcommand{\socialmediabutton}[2]{
  \button{#1}{#2}{\HUGE}\tiny
}

% Renders an extenral link button.
% param 1: link
\newcommand{\externallinkbutton}[1]{
  \button{#1}{\faExternalLinkSquare}{\large}\tiny
}

% Renders a progress-bar to indicate a certain skill in percent.
% param 1: name of the skill / tech / etc.
% param 2: level (for example in years)
% param 3: percent, values range from 0 to 1
\newcommand{\cvskill}[3] {
  \indent
	\begin{tabular*}{1\mpwidth}{p{0.63\mpwidth}  r}
    \textcolor{dark}{\textbf{#1}} & \textcolor{green}{#2}
	\end{tabular*}
  \\[-0.1cm]
  \indent
	\begin{tabular}{r}
    \begin{tikzpicture}[scale=1,rounded corners=2.1pt,very thin]
      \fill [light] (0,0) rectangle (1\mpwidth, 0.15);
      \fill [green] (0,0) rectangle (#3\mpwidth, 0.15);
    \end{tikzpicture}%
  \end{tabular}
  \\[0.6cm]
}

% Renders a table and a paragraph (cvtext) wrapped in a parbox (to ensure minimum content
% is glued together when a pagebreak appears).
% Additional Information can be passed in text or list form (or other environments).
% the work you did
% param 1: time-frame i.e. Sep 14 - Jan 15 etc.
% param 2: event name (job position etc.)
% param 3: Customer, Employer, Industry
% param 4: Short description
% param 5: Notes
\newcommand{\cvevent}[5] {
	% we wrap this part in a parbox, so title and description are not separated on a pagebreak
	% if you need more control on page breaks, remove the parbox
	\indent
  \parbox{\mpwidth}{
		\begin{tabular*}{1\mpwidth}{p{0.69\mpwidth}  r}
      \textcolor{dark}{\textbf{#2}} &
      \colorbox{green}{\makebox[0.27\mpwidth]{\textcolor{light}{#1}}} \\
			\textcolor{green}{\textbf{#3}}
      % \textcolor{dark}{#4} & \\
      % \textit{#5}
      \ifthenelse{\isempty{#4}}{}{#4\\}
      \ifthenelse{\isempty{#5}}{}{\textit{#5}}
		\end{tabular*}
	}
	\\[0.6cm]
}

% Renders the title of the CV.
% param 1: full name
% param 2: occupation
\newcommand{\cvtitle}[2]{
  \begin{center}
    \huge\textbf{\textcolor{dark}{#1}}\large\\[0.3cm]
    \textcolor{green}{\rule{0.15\textwidth}{2.4pt}}\large\\[0.3cm]
    \large\textcolor{dark}{#2}\\[0.3cm]
    % \textcolor{green}{\rule{0.15\textwidth}{2.4pt}}\\[0.3cm]
  \end{center}
  % \textcolor{green}{\rule{\mpwidth}{6pt}}
}

\newcommand{\selfie}{
  \includegraphics[width=\mpwidth]{selfie.png}
}

% remove page numbering
\pagenumbering{gobble}

\begin{document}
\columnratio{0.36}
\setlength{\columnsep}{3em}
\setlength{\columnseprule}{6pt}
\colseprulecolor{light}
\begin{paracol}{2}
\begin{leftcolumn}

\cvtitle{Μιχαήλ Ανάργυρος Ζαμάγιας}{Προπτυχιακός Σπουδαστής}
\vfill\null

\selfie
\vfill\null

\cvsection{Επικοινωνία}
\indent
\begin{tabular}{r}
  \socialmediabutton{mailto:mzamagias@gmail.com}{\faEnvelopeSquare}
  \socialmediabutton{https://github.com/mzamayias}{\faGithubSquare}
  \socialmediabutton{https://linkedin.com/in/mzamayias/}{\faLinkedinSquare}
  \socialmediabutton{tel:+306942422485}{\faPhoneSquare}
\end{tabular}
\vfill\null

\cvsection{Γνώσεις}
\cvskill{Python}{5+ χρόνια}{1}
\cvskill{Anaconda}{5+ χρόνια}{1}
\cvskill{Dart}{3+ χρόνια}{0.6}
\cvskill{Flutter}{3+ χρόνια}{0.6}
\cvskill{JS/TS, HTML, CSS}{1+ χρόνια}{0.2}
\cvskill{Web Development}{1+ χρόνια}{0.2}

% \vfill\null
% \cvsection{Γλώσσες}
% \cvskill{C}{3+ χρόνια}{0.6}
% \cvskill{Python}{5+ χρόνια}{1}
% \cvskill{Dart}{3+ χρόνια}{0.6}
% \cvskill{JS/TS, HTML, CSS}{<1 χρόνια}{0.2}

% \cvsection{Framworks}
% \cvskill{Anaconda}{4+ χρόνια}{0.8}
% \cvskill{Flutter}{3+ χρόνια}{0.6}
% \cvskill{Web Development}{1+ χρόνια}{0.2}

% \cvsection{Λειτουργικά Συστήματα}
% \cvskill{Windows}{3+ χρόνια}{0.6}
% \cvskill{Linx}{3+ χρόνια}{0.6}
% \cvskill{macOS}{1+ χρόνια}{0.2}
\end{leftcolumn}
\begin{rightcolumn}

\cvsection{Εκπαίδευση}
\cvevent
  {\textbf{2018 - παρών}}
  {Τμήμα Μηχανικών Πληροφορικής, ΤΕΙ Κρήτης}
  {Ηράκλειο, Ελλάδα}{}{}
\cvevent
  {\textbf{14/06/2018}}
  {Γενικό Λύκειο Κω "Οδυσσέας Ελύτης"}
  {Κως, Ελλάδα}
  {Βαθμός: Λίαν καλώς (17.3)}
  {\href{https://drive.google.com/file/d/1N7O_5-ysh0rsrD5t_r9FQwRoQxkAag6_/view}{Απολυτήριο Λυκείου}}
\cvevent
  {\textbf{23/05/2015}}
  {ECCE B2}
  {Κως, Ελλάδα}
  {Βαθμός: PASS}
  {\href{https://drive.google.com/file/d/1I111gvvsEiY88U4-dooUpVSacpRH6nDA/view}{Πιστοποιητικό}, \href{https://drive.google.com/file/d/1NZyxrZrtbMhuUdYUVoCGei3wt2vjQEja/view}{Αναφορά}}
\vfill\null

% \cvsection{Εργασιακή Εμπειρία}
% \cvevent
%   {\textbf{08/2019}}
%   {ΞΕΝΟΔΟΧΕΙΑΚΑΙ ΚΑΙ ΤΟΥΡΙΣΤΙΚΑΙ ΕΠΙΧΕΙΡΗΣΕΙΣ Κ ΜΗΤΣΗΣ ΑΕ}
%   {Κως, Ελλάδα}
%   {Βοηθός Σερβιτόρου \newline
%   \externallinkbutton{https://drive.google.com/file/d/1-TOhrADrFBGMzvOsNsj5AWfALbk_cBPh/view?usp=sharing}
%   }
% \vfill\null

\cvsection{Projects}
\cvevent
  {\textbf{2021}}
  {\href{https://github.com/mzamayias/flutter_ecommerce_website_demo}{flutter\_ecommerce\_website\_demo}}
  {Dart, Flutter}
  {Ιστοσελίδα (διαδικτυακού εμπορίου). Το fronend είναι κατασκευασμένο με Flutter και το backend με Firebase.}
  {Το project είναι μερικώς ολοκλημένο και ανενεργό, αλλά η ιστοσελίδα είναι διαθέσιμη \href{https://flutter-ecommerce-website-demo.web.app/}{εδώ}.}
\cvevent
  {\textbf{2021}}
  {\href{https://github.com/mzamayias/qrapp}{qrapp}}
  {Dart, Flutter}
  {Android και iOS εφαρμογή. Η εφαρμογή προσφέρει ανάγνωση και δημιουργεία QR κωδικών και το ιστορικό αυτών.}
  {Το project δεν είναι ολοκλημένο και η εφαρμογή δεν είναι διαθέσιμη στο διαδίκτυο.}
\cvevent
  {\textbf{2020}}
  {\href{https://github.com/mzamayias/game_2048}{game\_2048}}
  {Dart, Flutter}
  {Android και iOS εφαρμογή. Η εφαρμογή είναι υλοποίηση του παιχνιδιού \href{https://play2048.co}{2048}.}
  {Το project δεν είναι ολοκλημένο και η εφαρμογή δεν είναι διαθέσιμη στο διαδίκτυο.}
\cvevent
  {\textbf{2020}}
  {\href{https://github.com/mzamayias/ehealth_services_semester_project}{ehealth\_services\_semester\_project}}
  {Dart, Flutter}
  {Android και iOS εφαρμογή. Η εφαρμογή οπτικοποιεί δεδομένα ύπνου, καρδιακού παλμού, βημάτων και θερμίδων από fitness tracker.}
  {Το project δεν είναι ολοκλημένο και η εφαρμογή δεν είναι διαθέσιμη στο διαδίκτυο.}

\newpage

\end{rightcolumn}
\end{paracol}
\end{document}